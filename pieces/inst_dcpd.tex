%inst_dcpd.tex
\newcommand{\DCPDInstructionsA}{%
\begin{small}
	\textcolor{red}{%
(※)本行を含め、以下の赤字で記した説明文は申請書を作成する際には消去してください。\\
  (\texttt{\textbackslash DCPDInstructionsA}をコメントアウトしてください。)\\
・下記(1)及び(2)の記入にあたっては、例えば、研究における主体性、発想力、問題解決力、知識の幅・深さ、\\
 技量、コミュニケーション力、プレゼンテーション力などの観点から、具体的に記入してください。\\
 また、観点を項目立てするなど、適宜工夫して記入してください。\\
 なお、研究中断のために生じた研究への影響について、特筆すべき点がある場合には記入してください。
	}
	\end{small}
}

\newcommand{\DCPDInstructionsB}{%
\\
\begin{small}
	\textcolor{red}{%
(※)本行を含め、以下の赤字で記した説明文は申請書を作成する際には消去してください。\\
  (\texttt{\textbackslash DCPDInstructionsB}をコメントアウトしてください。)\\
・記述の根拠となるこれまでの研究活動の成果物(論文等)も適宜示しながら強みを記入してください。\\
%\begin{small}
 成果物(論文等)を記入する場合は、それらを同定するに十分な情報を記入してください。\\
	}
\end{small}
\begin{footnotesize}
	\textcolor{red}{%
(例)学術論文(査読の有無を明らかにしてください。査読のある場合、採録決定済のものに限ります。)\\
   著者、題名、掲載誌名、巻号、pp開始頁--最終頁、発行年を記載してください。\\
 (例) 研究発表(口頭・ポスターの別、査読の有無を明らかにしてください。)\\
   著者、題名、発表した学会名、論文等の番号、場所、月・年を記載してください。(発表予定のものは除く。\\
   ただし、発表申し込みが受理されたものは記載してもよい。)
	}
\end{footnotesize}
}
