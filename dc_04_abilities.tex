%#Split: 04_abilities  
%#PieceName: p04_abilities
% p04_abilities_00.tex
\KLBeginSubjectWithHeaderCommands{04}{4}{研究遂行力の自己分析}{2}{F}{}{\DCPDFirstSubjectPageStyle}{\DCPDDefaultPageStyle}

\section{研究遂行力の自己分析}
%    <<最大 2ページ>>

% s14_abilities
%begin 自己分析 ====================
% \DCPDInstructionsA\\% <-- 留意事項:これは消すか、コメントアウトしてください。
\noindent\textbf{(1) 研究に関する自身の強み}
% \DCPDInstructionsB% <-- 留意事項:これは消すか、コメントアウトしてください。

\noindent・\textbf{知識の幅・深さ、技量}

本申請を実施するうえで応募者の最大の強みと言えるのは、使用するソフトウェアの多くを自分で開発・保守を行ってきていることである。
中核となる準同型暗号ライブラリはCPU・GPU・FPGA用の3種類を開発しており、それらを用い論理回路を評価する実行エンジン、暗号上で動かすプロセッサまで研究に必要な幅広いソフトウェアをカバーしている~[\ref{code:vsp}]。
特に準同型暗号ライブラリの実装については情報処理推進機構のセキュリティキャンプにて講師として招聘された他~[\ref{achieve:seccamp}]、企業でのパートタイムジョブとしても行いその成果を査読付き国際ワークショップで発表する~[\ref{paper:wahc}]など、高い水準にある。
学部2,3年の時にはNHK学生ロボコンにて優勝~[\ref{achieve:nhk}]、専門分野を超えたハード・ソフトを問わない技量を養った。

\noindent・\textbf{研究における主体性}

申請者は高校生で最初の論文~[\ref{paper:rfid}]を出版しており、このときから研究との向き合い方を学んできた。
特に本申請の研究計画の核となる準同型暗号に関するテーマは、申請者は学部3年生の時点から継続して3年以上取り組んできた。
学部3年生のときに発案したテーマは、情報処理推進機構の未踏事業に採択、スーパクリエータとして認定され~[\ref{achieve:mitou}]、京都大学総長賞も受賞~[\ref{achieve:vsp}]、成果は査読付き国際学会に採択~[\ref{paper:usenix}]された。
研究室の継続的指導のない状態で研究を行えたことは主体性の証左であると考える。

\noindent・\textbf{発想力・問題解決能力}

学部2回生で準同型暗号を知り、もし準同型暗号がNANDを計算できるのであれば、暗号上でプロセッサが評価できると考えた。
最初に発案した準同型暗号のテーマ~[\ref{paper:usenix}]から始まり、本申請に至るまで、このアイデアに基づいており、プログラムとデータの両方を暗号によって保護するセキュリティと通常のCPUのような使い勝手という利便性をより広い条件で達成するべく進めている。
暗号上でのメモリ評価の高速化~[\ref{paper:usenix}]や、より複雑な論理ゲートの実現~[\ref{paper:wahc}]など、暗号上でのCPUの評価における問題をこれまで解決してきている。
また、準同型暗号の応用という面では、プロセッサの評価以外にもオートマトン~[\ref{paper:cav}]やBinalized Neural Network~[\ref{achieve:bnn}]なども試みており、分野をまたいだアイデアの実現に取り組んできた。
長期的な研究につながるアイデアと分野をまたいだ応用のアイデアに取り組み、その実現のために様々な問題を解決してきたことは申請者の高い発想力と問題解決能力の証と考える。

\noindent・\textbf{コミュニケーション力}
% 学外の人との関わり

未踏に採択されたチームは異なる背景を持った現在の所属研究室も異なる人間であり、技術的意見の相違から衝突することもままあるが、申請時点まで共同での研究を維持できている。
また、セキュリティキャンプでは毎年異なる学生に少人数でのゼミを行ってきた。
これらは申請者のコミュニケーション能力の傍証と考える。

\noindent・\textbf{プレゼンテーション力}

申請者は招待講演を含む国内外の会議で発表実績があり~[\ref{achieve:wcis}]、対外的な発表能力も身につけている。

\noindent\textbf{成果-レター誌・査読あり}
\begin{enumerate}[leftmargin=0.5cm]
    \setlength{\parskip}{0cm} % 段落間
    \setlength{\itemsep}{0cm} % 項目間
	\paper{An RFID tag identification protocol via Boolean compressed sensing,}{Kotaro Matsuoka \etal}{IEICE Communications Express, Volume 5, Issue 5}{}{118-123}{2016}\label{paper:rfid}
\end{enumerate}

\noindent\textbf{成果-国際学会またはワークショップでの発表・口頭・査読あり}
\begin{enumerate}[leftmargin=0.5cm]
    \setcounter{enumi}{1}
    \setlength{\parskip}{0cm} % 段落間
    \setlength{\itemsep}{0cm} % 項目間
    \paper{Virtual Secure Platform: A {Five-Stage} Pipeline Processor over {TFHE}}{Kotaro Matsuoka, \etal}{30th USENIX Security Symposium (USENIX Security 21)}{}{4007--4024}{Online, 2021年月}\label{paper:usenix}
    \paper{Oblivious Online Monitoring for Safety LTL Specification via Fully Homomorphic Encryption}{Ryotaro Banno, Kotaro Matsuoka \etal}{Computer Aided Verification}{}{}{2022年8月}\label{paper:cav}
    \paper{Towards Better Standard Cell Library: Optimizing Compound Logic Gates for TFHE
Association for Computing Machinery}{Kotaro Matsuoka, \etal}{In Proceedings of the
9th on Workshop on Encrypted Computing \& Applied Homomorphic Cryptography (WAHC '21), Association for Computing Machinery (ACM)}{}{pp.63–68}{Online, 2021年11月}\label{paper:wahc}
\end{enumerate}

\noindent\textbf{成果-国内学会またはシンポジウムでの発表・口頭・査読なし}
\begin{enumerate}[leftmargin=0.5cm]
    \setcounter{enumi}{4}
    \setlength{\parskip}{0cm} % 段落間
    \setlength{\itemsep}{0cm} % 項目間
     \item  完全準同型暗号における BNN を用いた高速な秘匿推論手法の実装と評価. 情報処理学会全国大会 橋詰陽太,古川修平,松本直樹,伴野良太郎,松岡航太郎,佐藤高史. 2022 年 3 月. オンライン.\label{achieve:bnn}
	\item  Virtual Secure Platform: A Five-Stage Pipeline Processor over TFHE (from Usenix Security 2021). 暗号と情報セキュリティワークショップ(WCIS 2021)招待講演, オンライン, 2021年9月 \label{achieve:wcis}
	\item  Virtual Secure Platform: A Five-Stage Pipeline Processor over TFHE. FIT2022 招待講演, オンライン, 2022年9月\label{achieve:fit}
\end{enumerate}


\noindent\textbf{成果-受賞}
\begin{enumerate}[leftmargin=0.5cm]
    \setcounter{enumi}{7}
    \setlength{\parskip}{0cm} % 段落間
    \setlength{\itemsep}{0cm} % 項目間
	\paper{2019 年度 IPA 未踏 IT 人材発掘・育成事業 スーパクリエータ「準同型暗号によるバーチャルセキュアプラットフォームの開発」}{松岡 航太郎}{\url{https://www.ipa.go.jp/files/000082597.pdf}}{}{}{2019}\label{achieve:mitou}
	\paper{令和元年度 京都大学総長賞 (NHK学生ロボコン2019~ABUアジア・太平洋ロボコン代表選考会~ 優勝、チェコ杯・NOK賞受賞、 ABUアジア・太平洋ロボコン選考会ベスト8、ナガセ賞、ベストデザイン賞受賞)}{京大機械研究会}{\url{https://www.kyoto-u.ac.jp/sites/default/files/embed/jaeducation-campusRecognitionpresidentsdocuments2019zyusyousyalist.pdf}}{}{}{2020}\label{achieve:nhk}
	\paper{令和 2 年度 京都大学総長賞}{松岡 航太郎 \etal}{\url{https://www.kyoto-u.ac.jp/sites/default/files/inline-files/r2-sochosho-jyusho-168da1573b39e3ca3fa2c6c362417307.pdf}}{}{}{2021}\label{achieve:vsp}
	\item CSS2021 優秀論文賞 \url{https://www.iwsec.org/css/2021/award.html} \label{achieve:css}
\end{enumerate}


\noindent\textbf{成果-公開ソフトウェア}
\begin{enumerate}[leftmargin=0.5cm]
    \setcounter{enumi}{11}
    \setlength{\parskip}{0cm} % 段落間
    \setlength{\itemsep}{0cm} % 項目間
	\item Virtual Secure Platform, \url{https://github.com/virtualsecureplatform}\label{code:vsp}
\end{enumerate}

\noindent\textbf{成果-その他活動}

\begin{enumerate}[leftmargin=0.5cm]
    \setcounter{enumi}{12}
	\item セキュリティキャンプ 全国大会 2020-2022 L-2 暗号のまま計算しようゼミ 講師 及び 2022 Lトラックプロデューサ, \url{https://www.ipa.go.jp/jinzai/camp/2022/zenkoku2022_program_profile.html}\label{achieve:seccamp}
\end{enumerate}

\vspace{5mm}
\noindent\textbf{(2) 今後研究者として更なる発展のため必要と考えている要素}



\noindent\textbf{要素1: 研究成果を論文としてまとめる能力}

これまでの論文執筆では、指導教員等からの指導が不可欠であったと感じている。
より高い執筆能力を獲得することは論文の投稿にかかるコストを下げより多くの研究に取り組むことを可能にすると同時に、論文化を意識した研究計画を立てる助けになるものである。
よって、成果を論文としてまとめる能力は研究者としてさらなる発展のため必要と考えている。

\noindent\textbf{要素2:暗号学の基礎的知識}

ここでいう暗号学の基礎的知識というのは、2つの意味合いを持つ。
一つにはどういうものならば安全であるか、という安全性の前提に関する知識である。
準同型暗号は安全性よりもその機能に重点が置かれやすく、機能の改善のために既存の安全生証明とは異なる仮定を課すことがある。
そのような仮定が安全であるかを、拡張の提案と同時に議論できる能力の獲得はより広範な研究テーマに取り組むことを可能にするものと考えている。
もう一つの意味合いは、ブロックチェーンなどの準同型暗号以外の暗号技術への一定程度の理解である。
本申請でVerfifiable Computationの統合を目指すように、準同型暗号単体では達成できないセキュリティが存在する。
よって、どのような技術を組み合わせればより高い次元のセキュリティが達成できるかの指針を得るのに十分な程度の知識を持つことはより広範な研究アイデアの発案につながるものと考える。

%end 自己分析 ====================

% p04_abilities_01.tex
\KLEndSubject{F}


