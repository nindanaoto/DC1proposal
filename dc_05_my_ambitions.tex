%#Split: 05_my_ambitions  
%#PieceName: p05_my_ambitions
% p05_my_ambitions_00.tex
\KLBeginSubjectWithHeaderCommands{05}{5}{目指す研究者像等}{1}{F}{}{\DCPDFirstSubjectPageStyle}{\DCPDDefaultPageStyle}

\section{目指す研究者像等}
%    <<最大 1ページ>>

% s17_my_ambitions
\noindent
\textbf{(1)目指す研究者像 {\footnotesize ※目指す研究者像に向けて身に付けるべき資質も含め記入してください。}}

%begin 目指す研究者像 ====================
% 資質について書く
% 1.と海外にする

\noindent\textbf{\textcircled{1}高校時代の自分の助けになるような情報発信}

私が人生で初めて読んだ論文は後保範先生の博士論文であり、この論文で学んだ高速乗算法が準同型暗号の高速化に生きている。
この経験から、オープンアクセスの論文は誰もが入手できる専門的かつ信頼性の高い情報源であり、高校生が大学院レベルの知識を得ることも可能にする公益性の高いものだと信じている。
勿論、前提知識なしに論文を読み解くことは難しく、書籍やブログなど、それ以外の補助的な情報発信も必要である。
この自身の経験を次の世代も得られるよう、\amikake{\textbf{最先端の知見をオープンに発信}}していくことが目指すべき姿であると考えている。
具体的には、自らの論文はすべてオープンアクセスまたはプレプリントを公開し、セキュリティキャンプのような教育的な情報発信の機会には積極的に参加、SNSやブログなどでも補助的な情報を発信する。
また、論文はその時の最新の知識を提供するものであるが、自分の知識を体系化し後世に伝える\amikake{\textbf{書籍を執筆することが人生の目標の一つ}}である。

\noindent\textbf{\textcircled{2}幅広い専門性の獲得}

私の座右の銘の一つは「\amikake{\textbf{アイデアは自分の子供}}」である。
アイデアを実現できるかできないかというのは発案した者の双肩にかかるものであり、最終的に諦めるのだとしても最大限の努力を払ってからにすべきであると考える。
この座右の銘を研究者としてのあり方に適用するならば、自分の能力の及ぶ限り研究アイデアの実施を試みるということであり、できる限り\amikake{\textbf{独力で実施するための幅広い専門性を獲得}}することが目指すべき研究者像であると考える。
幅広い専門性が必要と考えるのは、研究アイデアは自分の専門分野にとどまるものとは限らないと認識しているためである。
実際、私が最初の準同型暗号に関連する研究テーマを思いついたのは、準同型暗号という名前と「暗号のまま計算できる」という性質だけを知ったときで、準同型暗号に関する知識はまったくなかった。
研究アイデアの実現に自分の専門分野の外の知識が必要なとき、その\amikake{\textbf{学習コストを支払うことをためらわない}}ことが目指すべき研究者像であると考える。

\noindent\textbf{\textcircled{3}オープンソース実装による再現性の確保}

良い研究の必須要件の一つは、再現性が十分に確保されていることだと考えている。
実験や評価の結果が再現できなければ、ハードウェアの進歩などで環境が変わった場合に公正な比較ができない。
そのため、研究で作成したソフトウェア・ハードウェアは\amikake{\textbf{オープンソースにし後世の研究で自由に使えるように}}することが望ましい研究者像だと考えている。

% \noindent\textbf{2.2 単独で研究を続行できる幅広い専門性の獲得}

% 専門外の分野に踏み出す場合などは、他者の助力を請う事は多くの場合望ましいことである。
% しかし。研究の実施を他者に依存しきると、何らかの理由で助力を得ることが難しくなったときに研究を継続できなくなる。
% よって、規模の縮小などは致し方ないが、そのような場合も研究を単独で遂行できるだけの幅広い専門性を研究の過程で獲得できることが目指す研究者像であると考える。
 %end 目指す研究者像 ====================

\vspace{5mm}
\noindent
\textbf{(2)上記の「目指す研究者像」に向けて、特別研究員の採用期間中に行う研究活動の位置づけ}

\noindent\textbf{\textcircled{1} 高校時代の自分の助けになるような情報発信}

採用中に執筆する\amikake{\textbf{論文はすべてオープンアクセスにするか、プレプリント}}を公開する。
また、研究内容をTwitterで発信したり、Qiitaで基礎的な知識の解説を研究活動の一環として行いたいと考えている。

\noindent\textbf{\textcircled{2} 幅広い専門性の獲得}

本申請の研究計画の範囲はVerifiable Computationやプロセッサ設計など幅のあるものではあるが、ある程度基礎的な知識を得ている範囲で計画している。
これらに加え、エフォートの一部をブロックチェーンや疑似乱数などの暗号学の中の他分野や、機械学習などの応用としての他分野を学習することに当てたいと考えている。

\noindent\textbf{\textcircled{3} オープンソース実装による再現性の確保}

採用中に開発するハードウェア・ソフトウェアはApacheまたはGPL系のライセンスの元でオープンソースとして公開する。
実行環境もパブリッククラウドを利用するなど可能な範囲で再現性に配慮する。

%begin 研究活動の位置づけ ====================
% 実際に身につけるための活動を書く
% 論文のオープン化
% 他研究室との共同研究



%end 研究活動の位置づけ ====================

% p05_my_ambitions_01.tex
\KLEndSubject{F}


