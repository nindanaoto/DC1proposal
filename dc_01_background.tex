%#Split: 01_background  
%#PieceName: p01_background
% p01_background_00.tex
\KLBeginSubjectWithHeaderCommands{01}{2}{研究の位置づけ}{1}{F}{3}{\DCPDVeryFirstPageStyle}{\DCPDDefaultPageStyle}

\section{研究の位置づけ}
%    <<最大 1ページ>>

%s03_background
%begin 本研究の着想に至った経緯など ====================
\textbf{研究計画の背景}

 近年、クラウドコンピューティングが普及している。これはクラウドベンダから計算資源をオンデマンドに借りる仕組みである。しかしこの仕組みは、クラウドベンダが攻撃者にならないという信用の元で成り立っている仕組みである。個人情報などのデータや営業秘密を含むプログラムなどを扱う場合、このような信用は前提とするべきでない。ベンダへの信用を前提としない場合、クラウドコンピューティングには2つの問題がある。(i)データとプログラムがクラウドベンダに見えることと、(ii)実行結果がプログラムの出力だと保証できないことである。図1にこれらの問題をまとめた。これらは暗号学により解決可能[1,2]だが、そのままではクラウドコンピューティングのオンプレミスのサーバと同様に扱える利便性を損なってしまう。従って、これらの問題は(iii)利便性の維持をしたまま達成されなければならない。

\textbf{課題・分野の状況}
\begin{enumerate}
    \item データとプログラムがクラウドベンダに見える: 図1に示すように、現状のクラウドコンピューティングではサーバ上でデータとプログラムは復号される。準同型暗号を用いれば暗号化したまま計算を行えるが、webサービスのようにプログラム提供者とデータ提供者が分かれる場合、計算量が鍵の本数の2乗に比例する[1]という問題がある。
    \item 実行結果がプログラムの出力だと保証できない: クラウドベンダはコストを抑えるために、プログラムを実行せずに偽の結果を返す可能性がある。Verifiable Computation(VC)を用いれば結果がプログラムの実行結果であるかを検証できる。しかし、適用された準同型暗号が限られている[2]か、実装が存在しない[3]。
    \item 利便性の維持: 我々の過去の研究[4]では暗号化したままC言語を実行可能である。しかし、独自ISAを採用したためにコンパイラも独自であり、他の言語がサポートできていない。
\end{enumerate}

\textbf{着想に至った経緯}

準同型暗号を知った時、「これがNANDを計算できるのであれば、暗号のまま計算処理をするCPUが作れるはずだ」と考えた。本研究はこの発想を骨子として現代的コンピュータの進化形として構想した。
%end 本研究の着想に至った経緯など ====================

% p01_background_01.tex
\KLEndSubject{F}


